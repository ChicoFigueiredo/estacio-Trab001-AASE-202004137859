Oi
\subimport{fig/}{01_Teste}


\chapter[Lendo números grandes]{Lendo números grandes}
Números grandes - pib -
 4.063.152.801.861

\chapter[O Número]{O Número}

%=============================================================================
\section{História do Número}

Teste de Bibliografia \citet{Iezzi2019} com gosto de gás, visto \citet{Baldor2007Aritmetica} mencionar essa característica. \citet{Trajano1948} \nocite{Hernan1999}

%=============================================================================
\section{A contagem}

%=============================================================================
\section{A cardinalidade e a função contagem - o objeto número}

%=============================================================================
\section{Nomenclatura e simbologia dos números}

Vamos partir da premissa que você, leitor, saiba dos princípios básicos de:
\begin{itemize}[leftmargin=2cm]
    \item Contar elementos
    \item Medir a partir de um padrão
\end{itemize}

Com isso, no seu dia a dia você já faz operações de associação do concreto para o abstrato.\

A representação dos números nada mais é do que associar símbolos à essas ideias.\

%-----------------------------------------------------------------------------
\subsection{Representação numérica decimal ou algarismos na base 10}

Na aritmética se representam os números e operam os cálculos por meio de algarismos e a numeração é a parte da aritmética que cuida da escrita e expressão do número.\

No mundo ocidental, a numeração se baseia no fato de que desde criança contamos de 1 em 1, usando  os dedos da mão como referencial.

A numeração ela pode ser escrita, falada ou codificada.

%-----------------------------------------------------------------------------
\subsection{Numerais, no bom português}

São usados os seguintes símbolos ou cifras para representar qualquer número:

\begin{center}
    0 1 2 3 4 5 6 7 8 9
\end{center}

Esses símbolos, na ordem, são assim \textit{denominados}:

\begin{center}
    \begin{tabular}{|c|l|}
        \hline
        Símbolo & Nome \\
        \hline
        0 & zero \\
        \hline
        1 & um \\
        \hline
        2 & dois \\
        \hline
        3 & três \\
        \hline
        4 & quatro \\
        \hline
        5 & cinco \\
        \hline
        6 & seis \\
        \hline
        7 & sete \\
        \hline
        8 & oito \\
        \hline
        9 & nove \\
        \hline
    \end{tabular}
\end{center}

O zero (ou 0) é um símbolo que representa a ausência, a falta, o vazio.

Os demais numerais representam, nessa ordem apresentada, os 9 primeiros números sequenciais, a aplicação repetitiva do operador \textbf{seq}.

\begin{center}
    \begin{tabular}{|c|l|l|}
        \hline
        Símbolo & Nome & Obtenção\\ \hline
        0 & zero   &                \\ \hline
        1 & um     & seq(0) = 0 + 1 \\ \hline
        2 & dois   & seq(1) = 1 + 1 \\ \hline
        3 & três   & seq(2) = 2 + 1 \\ \hline
        4 & quatro & seq(3) = 3 + 1 \\ \hline
        5 & cinco  & seq(4) = 4 + 1 \\ \hline
        6 & seis   & seq(5) = 5 + 1 \\ \hline
        7 & sete   & seq(6) = 6 + 1 \\ \hline
        8 & oito   & seq(7) = 7 + 1 \\ \hline
        9 & nove   & seq(8) = 8 + 1 \\ \hline
    \end{tabular}
\end{center}


%=============================================================================
%\section{Algarismos romanos}