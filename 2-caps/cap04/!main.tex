\chapter{Resultados e Conclusão}

\section{Resultados}

De forma piegas, \citeaa{Balieiro2014} descreve com maestria: Não precisamos fazer nenhum esforço para constatarmos a existência de uma enraizada desigualdade socioeconômica em nossa sociedade. Basta darmos
uma volta pela cidade para percebermos as diversas formas – de sutis a escancaradas
– que dão tom às relações socioeconômicas que caracterizam a sociabilidade
no mundo capitalista. Mas como se exprimem essas desigualdades?
Em termos de possibilidades de aquisição material, asseguradas pela ocupação
profissional e pela origem social, que se refletem nas oportunidades de
estudo e de fruição de bens simbólicos. Quando pensamos em termos de escolhas
religiosas, de gênero (masculino e feminino), etnia, geração etc., devemos
mobilizar um outro conceito: o de diferença.

Acrescento que, após refletir sobre o tema, e fazer um esforço sobre-humano para limitar-me ao escopo
devo concluir que, apesar da bussola apontar uma direção, a de termos um tratamento melhor do
meio ambiente em que vivemos de forma mais racional, creio que estamos ainda longe.

A reflexão que fiz, pelo fato de estar em isolamento social a 70 dias, me dei conta da quantidade de lixo que produzo.
Enorme, já que, dependendo do dia, posso gerar 3 sacos de 100 litros com lixo.

Por orientação e iniciativa, fazemos uma reciclagem, separando orgânicos e molhados em um cesto e recicláveis secos em outro.

A reciclagem, apesar de estar mais evoluído aqui em brasília, ainda para mim não constitui \textbf{fato social} na perspectiva de Durkheim, com
COERÇÃO SOCIAL \footnote{Os fatos exercem uma força sobre os indivíduos, levando-os a
confrontarem-se com as regras da sociedade em que vivem, tanto que os indivíduos sofrem sanções ou punições quando se rebelam contra essas regras.}
EXTERIORIDADE AOS INDIVÍDUOS \footnote{Os fatos sociais independem das vontades individuais ou da adesão consciente a eles. As regras sociais de conduta, as leis e os costumes já existem quando o sujeito nasce e são impostos a ele pela educação.} e
GENERALIDADE \footnote{social todo fato que é geral, ou seja, que se repete em todos os indivíduos ou na maioria deles. As formas de habitação, de comunicação, os sentimentos e a moral são alguns exemplos. A generalidade do fato social garante sua normalidade, ou seja, sua aceitação pela coletividade.}.
Embora

\section{Conclusão}

A melhor descrição que a sociedade é um gigantesco barco, e que qualquer mudança de direção, ou comportamento, por menor que seja, não é manobrável rapidamente, exigindo pequenos movimentos precisos para mudar a direção.


Termino esse arrazoado de ideias desejando que se tornem realidade as palavras de Michael Jackson:

\begin{citacao}
    Heal the world\\
    Make it a better place\\
    For you and for me\\
    And the entire human race\\

    There are people dying\\
    If you care enough for the living\\
    Make it a better place\\
    For you and for me \footnote{
        Cure o mundo / Faça dele um lugar melhor / Para você e para mim / E toda a raça humana //
        Há pessoas morrendo / Se você se importa o suficiente com a vida / Faça dele um lugar melhor / Para você e para mim \\
        Michael Jackson, Heal the World, Dangerous (1991)
    }
\end{citacao}
