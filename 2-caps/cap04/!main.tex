\chapter{Resultados e Conclusão}

\section{Resultados}

De forma piegas, \citeaa{Balieiro2014} descreve com maestria: Não precisamos fazer nenhum esforço para constatarmos a existência de uma enraizada desigualdade socioeconômica em nossa sociedade. Basta darmos
uma volta pela cidade para percebermos as diversas formas – de sutis a escancaradas
– que dão tom às relações socioeconômicas que caracterizam a sociabilidade
no mundo capitalista. Mas como se exprimem essas desigualdades?
Em termos de possibilidades de aquisição material, asseguradas pela ocupação
profissional e pela origem social, que se refletem nas oportunidades de
estudo e de fruição de bens simbólicos. Quando pensamos em termos de escolhas
religiosas, de gênero (masculino e feminino), etnia, geração etc., devemos
mobilizar um outro conceito: o de diferença.

Acrescento que, após refletir sobre o tema, e fazer um esforço sobre-humano para limitar-me ao escopo
devo concluir que, apesar da bussola apontar uma direção, a de termos um tratamento melhor do
meio ambiente em que vivemos de forma mais racional, creio que estamos ainda longe.

A reflexão que fiz, pelo fato de estar em isolamento social a 70 dias, me dei conta da quantidade de lixo que produzo.
Enorme, já que, dependendo do dia, posso gerar 3 sacos de 100 litros com lixo.

Por orientação e iniciativa, fazemos uma reciclagem, separando orgânicos e molhados em um cesto e recicláveis secos em outro.

A reciclagem, apesar de estar mais evoluído aqui em brasília do que em outros municípios, ainda para mim não constitui \textbf{fato social} na perspectiva de Durkheim, com
\textit{coerção social} \footnote{Os fatos exercem uma força sobre os indivíduos, levando-os a
confrontarem-se com as regras da sociedade em que vivem, tanto que os indivíduos sofrem sanções ou punições quando se rebelam contra essas regras.},
\textit{exterioridade aos indivíduos} \footnote{Os fatos sociais independem das vontades individuais ou da adesão consciente a eles. As regras sociais de conduta, as leis e os costumes já existem quando o sujeito nasce e são impostos a ele pela educação.} e
\textit{generalidade} \footnote{social todo fato que é geral, ou seja, que se repete em todos os indivíduos ou na maioria deles. As formas de habitação, de comunicação, os sentimentos e a moral são alguns exemplos. A generalidade do fato social garante sua normalidade, ou seja, sua aceitação pela coletividade.}.

Embora reconheça que essas característica estejam ganhando corpo, ainda é fácil macular essas premissas, basta ver as quantidades absurdas de lixo que surge ao final de um show ou festa pública. Jogar um papel na rua não provoca indignação ou multa, embora possam existir tais leis.
Ainda sim, cansei de ver lixo dispensado de forma errada, misturando os dois tipos, ou mesmo dispensado na lata errada.

Eu penso se a inabilidade ou displicência das pessoas não seja um \textbf{fato social} inconsciente na mente coletiva. Maus hábitos culturalmente imposto ou passado por osmose de pai para filho.
Nessa horá só fico me martelando o vídeo do professor \citeaa{Cortella2014} no Youtube\footnote{disponível em \url{https://www.youtube.com/watch?v=2gVCs2fIILo}}: A ética é um conjunto de princípios e valores que você usa para responder às três grandes perguntas da vida humana: Quero? Devo? Posso?

Nós vivemos muitas vezes dilemas éticos. Há coisas que eu quero, mas não devo. Há coisas que eu devo, mas não posso. Há coisas que eu posso, mas não quero. Quando você tem paz de espírito? Quando tem um pouco de felicidade? Quando aquilo que você quer é o que você deve e o que você pode. Todas as vezes que aquilo que você quer não é aquilo que você deve; todas as vezes que aquilo que você deve não é o que você pode; todas as vezes que aquilo que você pode não é o que você quer, você vive um conflito, que muitas vezes é um dilema.

Logicamente quando não reciclo, entro em um dilema ético na hora ou em algum momento posterior, advindo do fato de que pesará a consciência. Quero reciclar, devo reciclar e posso reciclar. Quebrar isso me produz arrependimento e culpa.

O fato é: reciclar ainda é por "adesão" voluntária, deixar de reciclar ou sujar a rua não faz com que a maioria das pessoas não saiam de uma certa zona de conforto ao não faze-lo, mesmo nas mas simples práticas de jogar um papel de bala no chão me leva a pensar na bagagem cultural que temos.

Persigo a autoridade do exemplo, por que é no exemplo que podemos ensinar as pessoas. Isso vale para praticamente tudo. Quando criança tinha que os policiais seriam a "autoridade máxima do exemplo", que pagariam tudo em dia, teriam boas notas, teriam autoridades de como agir, falar, pois seriam exemplos. Não foi inúmeras decepções com milicianos e padres pedófilos.

Não sou o melhor exemplo do mundo, mas ao tomar consciência do meu papel para um mundo melhor, corrigir erros, eu do o exemplo, mesmo que solitário no meu apartamento. Pois o exemplo vem do hábito, da disciplina inconsciente do hábito. Eu já joguei muito lixo na rua, mas hoje, ao tentar faze-lo, abre-se um gatilho e meu mundo desaba, a ponto de as vezes depois de vários passos voltar, pegar e levar para a lata de lixo mais próxima, ou para casa.

\subsection{Marx e o lixo}

Curiosamente, durante minhas reflexões acerca da destinação do lixo, me deu um insight de como as desigualdades tão diametralmente opostas entre o pontão e o lixão, dos catadores trabalhando no lixo e da luta do novo aterro, não deixei de ter "que" marxista no que posso chamar da luta de classes entre o "proletariado" catador de lixo e a "burguesia" estatal na luta que culminou na batalha para migração do lixão da estrutural para o aterro sanitário.

Sem deméritos da profissão de catador, ao contrário, bendito seja o lixo da rica burguesia que gera renda para os menos afortunados. Triste pensar que a falta de emprego possa gerar a situação de centenas de pessoas, as vezes jovens e crianças, a depender das migalhas putrefatas do lixo para conseguir uns trocados e conseguir alimentação.




\subsection{A escola classe no parque do lixo}

Ao mesmo tempo, me alegra ver as iniciativas no âmbito da educação ambiental, pessoas que tem, por atividades práticas e in loco, contato tanto com a natureza e suas belezas quanto ao lixo que é produzido pela sociedade.
Conhecer o padrão introduzido pelo Anísio Teixeira sobre a educação na nova capital mostra que existem iniciativas, embora solitárias, de querer mudar o mundo por meio da educação.

Confesso que ao analisar a tabela~\ref{table:IDH} me salta aos olhos a uniformidade do IDH-E\footnote{Indice de Desenvolvimento Humano Educacional}, com exceção de Planaltina, todas estão acima de 0,9.
Essa uniformidade me parece transparecer, numa análise rasa e superficial, que o planejamento da educação parece estar colhendo alguns frutos, a despeito da distribuição da renda ser um tanto desigual, como demonstra realmente a tabela~\ref{table:IDH} na sua coluna IDH-R\footnote{Indice de Desenvolvimento Humano Educacional}.

Ainda sim, penso que apesar dos esforços, do trabalho de campo, ainda existe uma cultura muito arraigada. Ainda mais com a diferença de classes sociais muito explícitos entre o aluno do colégio militar e o aluno da rede pública. Olhando os microdados do ENEM da página do \citeaa{INEP2019} com ferramentas de Data Science notamos que brasilia é a melhor na nota combinada de português, matemática e redação.

% Please add the following required packages to your document preamble:
% \usepackage[table,xcdraw]{xcolor}
% If you use beamer only pass "xcolor=table" option, i.e. \documentclass[xcolor=table]{beamer}
% \usepackage[normalem]{ulem}
% \useunder{\uline}{\ul}{}
\begin{table}[]
    \begin{tabular}{lllllllllll}
        &
        &
        &
        \multicolumn{2}{c}{\cellcolor[HTML]{4F81BD}{\color[HTML]{FFFFFF} Matemática}} &
        \multicolumn{2}{c}{\cellcolor[HTML]{4F81BD}{\color[HTML]{FFFFFF} Português}} &
        \multicolumn{2}{c}{\cellcolor[HTML]{4F81BD}{\color[HTML]{FFFFFF} Redação}} &
        \multicolumn{2}{c}{\cellcolor[HTML]{4F81BD}{\color[HTML]{FFFFFF} Geral}} \\
        \rowcolor[HTML]{4F81BD}
        \multicolumn{1}{c}{\cellcolor[HTML]{4F81BD}{\color[HTML]{FFFFFF} UF}} &
        \multicolumn{1}{c}{\cellcolor[HTML]{4F81BD}{\color[HTML]{FFFFFF} Nome}} &
        \multicolumn{1}{c}{\cellcolor[HTML]{4F81BD}{\color[HTML]{FFFFFF} n}} &
        \multicolumn{1}{c}{\cellcolor[HTML]{4F81BD}{\color[HTML]{FFFFFF} média ($\mu$)}} &
        \multicolumn{1}{c}{\cellcolor[HTML]{4F81BD}{\color[HTML]{FFFFFF} desvio ($\sigma$)}} &
        \multicolumn{1}{c}{\cellcolor[HTML]{4F81BD}{\color[HTML]{FFFFFF} média ($\mu$)}} &
        \multicolumn{1}{c}{\cellcolor[HTML]{4F81BD}{\color[HTML]{FFFFFF} desvio ($\sigma$)}} &
        \multicolumn{1}{c}{\cellcolor[HTML]{4F81BD}{\color[HTML]{FFFFFF} média ($\mu$)}} &
        \multicolumn{1}{c}{\cellcolor[HTML]{4F81BD}{\color[HTML]{FFFFFF} desvio ($\sigma$)}} &
        \multicolumn{1}{c}{\cellcolor[HTML]{4F81BD}{\color[HTML]{FFFFFF} média ($\mu$)}} &
        \multicolumn{1}{c}{\cellcolor[HTML]{4F81BD}{\color[HTML]{FFFFFF} desvio ($\sigma$)}} \\
        53 & Distrito Federal    & 59.574  & 5,44 & 1,10 & 6,66 & 0,87 & 5,53 & 1,74 & 5,87 & 1,04 \\
        \rowcolor[HTML]{DCE6F1}
        33 & Rio de Janeiro      & 232.620 & 5,49 & 1,13 & 6,59 & 0,86 & 5,54 & 1,93 & 5,86 & 1,11 \\
        31 & Minas Gerais        & 315.515 & 5,51 & 1,12 & 6,52 & 0,85 & 5,48 & 1,92 & 5,82 & 1,11 \\
        \rowcolor[HTML]{DCE6F1}
        42 & Santa Catarina      & 80.684  & 5,54 & 1,06 & 6,60 & 0,83 & 5,35 & 1,71 & 5,82 & 1,01 \\
        35 & São Paulo           & 684.955 & 5,47 & 1,07 & 6,61 & 0,84 & 5,28 & 1,72 & 5,78 & 1,01 \\
        \rowcolor[HTML]{DCE6F1}
        43 & Rio Grande do Sul   & 127.848 & 5,38 & 1,01 & 6,55 & 0,82 & 5,33 & 1,74 & 5,74 & 0,98 \\
        41 & Paraná              & 168.753 & 5,38 & 1,05 & 6,53 & 0,83 & 5,17 & 1,71 & 5,69 & 0,99 \\
        \rowcolor[HTML]{DCE6F1}
        32 & Espírito Santo      & 70.163  & 5,39 & 1,10 & 6,41 & 0,86 & 5,28 & 1,98 & 5,68 & 1,12 \\
        28 & Sergipe             & 35.740  & 5,17 & 1,04 & 6,27 & 0,89 & 5,44 & 2,06 & 5,62 & 1,14 \\
        \rowcolor[HTML]{DCE6F1}
        24 & Rio Grande do Norte & 50.892  & 5,27 & 1,05 & 6,34 & 0,90 & 5,26 & 1,98 & 5,61 & 1,12 \\
        52 & Goiás               & 117.727 & 5,26 & 1,03 & 6,35 & 0,86 & 5,22 & 1,92 & 5,60 & 1,09 \\
        \rowcolor[HTML]{DCE6F1}
        25 & Paraíba             & 60.205  & 5,19 & 1,02 & 6,24 & 0,87 & 5,29 & 2,01 & 5,57 & 1,10 \\
        50 & Mato Grosso do Sul  & 39.974  & 5,23 & 1,02 & 6,35 & 0,87 & 5,12 & 1,90 & 5,56 & 1,08 \\
        \rowcolor[HTML]{DCE6F1}
        29 & Bahia               & 186.330 & 5,10 & 0,99 & 6,23 & 0,88 & 5,15 & 1,90 & 5,48 & 1,06 \\
        26 & Pernambuco          & 162.492 & 5,21 & 1,02 & 6,24 & 0,87 & 5,02 & 2,02 & 5,48 & 1,11 \\
        \rowcolor[HTML]{DCE6F1}
        22 & Piauí               & 63.574  & 5,08 & 1,03 & 6,12 & 0,90 & 5,09 & 2,06 & 5,42 & 1,14 \\
        51 & Mato Grosso         & 54.855  & 5,13 & 0,96 & 6,23 & 0,84 & 4,95 & 1,97 & 5,42 & 1,06 \\
        \rowcolor[HTML]{DCE6F1}
        27 & Alagoas             & 46.462  & 5,06 & 0,97 & 6,13 & 0,86 & 5,01 & 2,03 & 5,39 & 1,08 \\
        15 & Pará                & 135.536 & 4,98 & 0,91 & 6,10 & 0,86 & 5,05 & 2,01 & 5,37 & 1,06 \\
        \rowcolor[HTML]{DCE6F1}
        23 & Ceará               & 240.684 & 5,10 & 1,01 & 6,12 & 0,88 & 4,87 & 2,19 & 5,35 & 1,18 \\
        14 & Roraima             & 8.747   & 5,06 & 0,90 & 6,21 & 0,85 & 4,76 & 1,87 & 5,33 & 0,99 \\
        \rowcolor[HTML]{DCE6F1}
        11 & Rondônia            & 32.834  & 5,04 & 0,89 & 6,16 & 0,81 & 4,82 & 1,90 & 5,33 & 1,00 \\
        17 & Tocantins           & 29.849  & 5,00 & 0,93 & 6,07 & 0,85 & 4,90 & 1,85 & 5,31 & 1,01 \\
        \rowcolor[HTML]{DCE6F1}
        12 & Acre                & 17.161  & 4,90 & 0,83 & 6,05 & 0,84 & 4,82 & 1,91 & 5,25 & 0,99 \\
        16 & Amapá               & 18.553  & 4,89 & 0,83 & 6,07 & 0,85 & 4,80 & 1,93 & 5,25 & 1,00 \\
        \rowcolor[HTML]{DCE6F1}
        13 & Amazonas            & 78.471  & 4,93 & 0,86 & 6,06 & 0,87 & 4,61 & 1,89 & 5,19 & 1,00 \\
        21 & Maranhão            & 114.537 & 4,89 & 0,85 & 6,00 & 0,84 & 4,68 & 1,96 & 5,18 & 1,01
    \end{tabular}
    \caption{}
    \label{table:IDH}
\end{table}



\section{Conclusão}

A melhor descrição que a sociedade é um gigantesco barco, e que qualquer mudança de direção, ou comportamento, por menor que seja, não é manobrável rapidamente, exigindo pequenos movimentos precisos para mudar a direção.


Termino esse arrazoado de ideias desejando que se tornem realidade as palavras de Michael Jackson:

\begin{citacao}
    Heal the world\\
    Make it a better place\\
    For you and for me\\
    And the entire human race\\

    There are people dying\\
    If you care enough for the living\\
    Make it a better place\\
    For you and for me \footnote{
        Cure o mundo / Faça dele um lugar melhor / Para você e para mim / E toda a raça humana //
        Há pessoas morrendo / Se você se importa o suficiente com a vida / Faça dele um lugar melhor / Para você e para mim \\
        Michael Jackson, Heal the World, Dangerous (1991)
    }
\end{citacao}
