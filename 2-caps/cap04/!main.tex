\chapter{Resultados e Conclusão}

\section{Resultados}

De forma piegas, \citeaa{Balieiro2014} descreve com maestria: Não precisamos fazer nenhum esforço para constatarmos a existência de uma
enraizada desigualdade socioeconômica em nossa sociedade. Basta darmos
uma volta pela cidade para percebermos as diversas formas – de sutis a escancaradas
– que dão tom às relações socioeconômicas que caracterizam a sociabilidade
no mundo capitalista. Mas como se exprimem essas desigualdades?
Em termos de possibilidades de aquisição material, asseguradas pela ocupação
profissional e pela origem social, que se refletem nas oportunidades de
estudo e de fruição de bens simbólicos. Quando pensamos em termos de escolhas
religiosas, de gênero (masculino e feminino), etnia, geração etc., devemos
mobilizar um outro conceito: o de diferença.

Acrescento que, após refletir sobre o tema,


\section{Conclusão}

Termino esse arrazoado de ideias desejando que se tornem realidade as palavras de Michael Jackson:

\begin{citacao}
    Heal the world\\
    Make it a better place\\
    For you and for me\\
    And the entire human race\\

    There are people dying\\
    If you care enough for the living\\
    Make it a better place\\
    For you and for me \footnote{
        Cure o mundo / Faça dele um lugar melhor / Para você e para mim / E toda a raça humana //
        Há pessoas morrendo / Se você se importa o suficiente com a vida / Faça dele um lugar melhor / Para você e para mim \\
        Michael Jackson, Heal the World, Dangerous (1991)
    }
\end{citacao}
