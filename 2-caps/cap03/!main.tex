\chapter{Procedimentos Metodológicos}

\section{Pesquisa}

Foi realizada intensa pesquisa on-line na internet por temas chaves como:
\begin{itemize}
    \item Brasília
    \item Distrito Federal
    \item Meio Ambiente
    \item Educação Ambiental
    \item Sociologia
    \item Sociologia Ambiental
    \item Reciclagem
\end{itemize}

Após breve leitura ou visualização de vídeos, foram selecionados os itens que mais se adequavam ao contexto e ao pensamento reflexivo para formação de um encadeamento de ideias, formando o rol de itens bibliográficos que se encontram descrito nas referências.

\section{Reflexão Sociológica}

Como instruído:
\begin{itemize}
    \item Aplicar de modo reflexivo paradigmas e conceitos estudados na disciplina para a construção de práticas de educação, que tornem mais equilibradas as relações entre a sociedade e o meio ambiente.
    \item Observar as diferentes interações estabelecidas pelas comunidades com os recursos naturais do entorno sociopolítico cultural onde vive e convive.
    \item Aplicar na prática conceitos da Sociologia e da Antropologia relacionados às práticas de educação e desenvolvidos pela disciplina Aspectos Antropológicos e Sociológicos da Educação, adotando postura de futuro (a) professor (a).
\end{itemize}

Coube avaliar os itens pesquisados e submete-los a reflexão sociológica.

Após a composição desse material foi montado uma apresentação e uma video-aula que estão, respectivamente, nos seguintes endereços:

Apresentação no OneDrive: \url{https://1drv.ms/p/s!AgRBucATAhUblzAldnG4LGnWNV-r?e=ykIvGi} \\
\qrset{link, height=4cm}\begin{center}
    \qrcode{https://1drv.ms/p/s!AgRBucATAhUblzAldnG4LGnWNV-r?e=ykIvGi}
\end{center}




Vídeo no YouTube: \url{https://1drv.ms/p/s!AgRBucATAhUblzAldnG4LGnWNV-r?e=ykIvGi} \\
\qrset{link, height=4cm}
\begin{center}
    \qrcode{https://1drv.ms/p/s!AgRBucATAhUblzAldnG4LGnWNV-r?e=ykIvGi}
\end{center}