\section{Meio ambiente e educação ambiental}

Segundo \citeaa{McReynolds1999}, A Sociologia do meio ambiente é uma área antiga, apesar de nova. Marx e Engels (\citeaa{Marx1961}), Weber e Durkheim (1954,1982), todos escreveram sobre o relacionamento entre sociedades humanas e o meio ambiente natural. Contudo, o termo "Sociologia do meio ambiente" não foi utilizado até 1971. Em 1976, A Sociedade Americana de Sociologia designou uma seção para a área. Em 1978, Catton e Dunlop publicaram a primeira tentativa de proporcionar uma definição explícita da área de sociologia do meio ambiente. E, não foi antes de 1990 que a Sociedade Internacional de Sociologia formou o seu primeiro grupo com interesse específico em sociologia do meio ambiente.\\

Hoje em dia, a sociologia do meio ambiente procura incorporar mais variáveis científicas naturais, perspectivas e até paradigmas em seus métodos, teorias e literatura. O aumento do crescimento e do interesse em perspectivas multi e interdisciplinares também acrescentou em amplitude para o aprofundamento da sociologia do meio ambiente. Esta expansão fez da sociologia ambiental um emaranhado de disciplinas com bases crescentes na biologia, ecologia, ciência política, antropologia, psicologia, feminismo e outras. Apesar da aparência pós-moderna, a sociologia do meio ambiente ainda pretende ser a única linha de pensamento viável capaz de proporcionar uma perspectiva macro ou além, conforme \citeaa{CATTON1978} e \citeaa{REDCLIFT1997}\\

A Sociologia do meio ambiente tem sido definida de diversas maneiras. Entre as várias definições, \citeaa{BUTTEL1996}  proporciona um começo útil. Ele nota que hoje em dia a essência da sociologia do meio ambiente tem sido de recuperar e revelar a materialidade da estrutura e vida social, e o faz de maneira a produzir entendimentos relevantes de modo a resolver problemas ambientais. Esta definição reconhece ao mesmo tempo a centralização da verdadeira natureza física do meio ambiente e o papel representado pelas construções sociais da natureza.\\

A união da natureza física e das construções sociais da natureza permanece atualmente como a principal preocupação para a sociologia ambiental. Na verdade, a habilidade de unir estes conceitos aparece como o centro da pretensão da área de ser a melhor das áreas da sociologia a se aplicar a um dos maiores problemas mundiais - o declínio do meio ambiente. Com o final da Guerra Fria, as preocupações sobre o aquecimento global e mudanças no meio ambiente mundial tomaram o lugar das preocupações com a guerra nuclear. Sendo assim, a sociologia do meio ambiente tem ocupado o cenário central na relação dos problemas mundiais [\citeaa{VAILLANCOURT1995}].\\

Neste contexto, a sociologia do meio ambiente está preocupada com uma vasta gama de questões, campos de estudo e disciplinas. Se por um lado essa amplitude é excitante, é fácil se perder no labirinto do que veio a se tornar a sociologia do meio ambiente. Nas páginas seguintes apresento uma bibliografia de fontes e uma lista de jornais que relevam o conhecimento da área. Nenhum deles pretende ser exaustivo. Ao contrário, eles têm a intenção de proporcionar ao estudante novo e intermediário da área um acesso mais direto à literatura histórica crítica, teórica e metodológica. Com essas bases, espera-se que o leitor fique mais preparado para pesquisar este crescente e importante campo da sociologia.\\

Já \citeaa{Dias1994}  diz que a "Educação Ambiental se caracteriza por
incorporar as dimensões sociais, políticas, econômicas,
culturais, ecológicas e éticas, deixando claro que ao discutir
qualquer problema ambiental é fundamental a consideração
de todos estes aspectos." Segundo este autor, "a maior parte
dos problemas ambientais tem suas raízes na miséria que,
por sua vez, é gerada por políticas e problemas econômicos,
concentradores de riqueza e responsáveis pelo desemprego
e degradação ambiental."\\

Pode-se também definir a educação ambiental, nas palavras de \citeonline{Magalhaes2018}, como um processo
onde o educando obtém conhecimentos acerca das
questões ambientais e assim passa a ter um novo
entendimento acerca do meio ambiente, se tornando um
agente transformador referente à preservação do meio
ambiente e de seus recursos naturais. \\

\citeaa{Gadotti2000} explica que educação ambiental vai muito além do conservacionismo
Trata-se de uma mudança radical de mentalidade em
relação à qualidade de vida, que está diretamente ligada
ao tipo de convivência que mantemos com a natureza e
que implica em atitudes, valores, ações. Trata-se de uma
opção de vida por uma relação saudável e equilibrada,
com o contexto, com os outros, com o ambiente mais
próximo, a começar pelo ambiente de trabalho e
doméstico.\\
