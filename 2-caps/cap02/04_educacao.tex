\section{Educação}

De acordo com \citeaa{Pereira2005}, a construção da nova capital do Brasil constituía-se em uma das metas da
política nacional-desenvolvimentista implementada pelo governo Juscelino
Kubitscheck. Brasília seria um ponto de germinação para o interior, visando à
integração entre centros urbanos e regiões agropecuárias, por meio de um complexo
rodoviário.\\

Para viabilizar esse empreendimento, foi criada, em 1956, a Companhia
Urbanizadora da Nova Capital do Brasil (NOVACAP), diretamente subordinada ao
Presidente da República. Além de responsabilizar-se pela construção de Brasília, essa
instituição encarregou-se de criar diversos organismos ou setores necessários ao
funcionamento da cidade. Em decorrência, criou-se, no final de 1956, o Departamento
de Educação e Saúde, mais tarde denominado Departamento de Educação e Difusão
Cultural, cuja finalidade era promover atividades educacionais, em caráter emergencial,
até a implantação definitiva do sistema educacional do Distrito Federal.\\

Em meados de 1957, com a chegada das primeiras famílias de operários e
funcionários ao Planalto Central, o número de crianças passou a ser uma preocupação
por parte do poder público, preocupação essa que aumentava na medida em que crescia
o fluxo migratório para Brasília.\\

Por iniciativa do Departamento de Educação e Difusão Cultural, foram criadas
as primeiras escolas provisórias da nova Capital. Para tanto, o referido Departamento,
sob a coordenação do médico Ernesto Silva, buscou assessoramento técnico junto ao
educador Anísio Teixeira, então diretor do INEP. Nessa ocasião, foi-lhe também
solicitada orientação geral sobre o sistema escolar da nova capital do País.
Em 1959, foi instituída, no Ministério da Educação e Cultura, a Comissão de
Administração do Sistema Educacional de Brasília (CASEB) \footnote{Ver, Decreto Presidencial n. 47.472, de 22 de novembro de 1959.} , tendo Anísio Teixeira
dela participado como membro da Comissão Deliberativa. Responsabilizando-se pela
elaboração do referido plano, o educador deu origem ao documento intitulado “Plano de
Construções Escolares de Brasília”, que veio a público em 1961, na Revista Brasileira
de Estudos Pedagógicos \footnote{Ver, Anísio Teixeira. Plano de Construções Escolares. Revista Brasileira de Estudos Pedagógicos n° 81, volume 35, jan/mar- 1961, p.195-199.}.

Ainda de acordo com \citeaa{Pereira2005} Anísio Teixeira propugnava por transformações educacionais que viabilizassem
a adequação do sistema de educação ao estado democrático moderno. Entre as suas
atribuições à frente do INEP, cabia-lhe a responsabilidade pela política e planejamento
educacional. Consciente, porém, das dificuldades que se sobrepunham às mudanças
preconizadas, em face da insuficiência de recursos econômicos, materiais e humanos,
propôs que as bases da reorganização institucional fossem inicialmente lançadas no
ensino primário, mediante a instalação de centros de demonstração, distribuídos pelas
diversas regiões do País. \\

Nessa perspectiva, não seria Brasília um locus ideal para a implantação da escola
renovada? O que significaria implantá-la numa cidade nova, moderna, a partir do nada
existente, sem as amarras da tradição? Que influência poderia exercer nos domínios da educação do País? Em que medida iria se refletir no sentido e direção das tendências do
ensino? \\
\begin{citacao}
    Tais preocupações parecem ter sido centrais no planejamento educacional da
    nova Capital. Na parte introdutória do plano, acha-se claramente explicitado que:
    O plano de construções escolares para Brasília obedeceu ao propósito de
    abrir oportunidade para a Capital Federal oferecer à Nação um conjunto de
    escolas que pudessem constituir exemplo e demonstração para o sistema
    educacional do País
\end{citacao}