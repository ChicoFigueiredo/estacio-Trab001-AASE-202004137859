\chapter{Objetivos}

O presente trabalho visa ensaiar um estudo sociológico de Brasília sob a ótica da relação entre a sociedade e o meio ambiente, discutindo a importância da educação como fomentador para a conscientização e evolução do pensamento relacionado aos impactos ambientais causados pelo homem, em especial delimitado no quadrado geográfico incrustado no meio do planalto central. \\

Além de ser essencial para o desenvolvimento intelectual dos alunos a partir da reflexão sobre a importância da natureza, dos impactos diretos sobre a ação humana. \\

O trabalho será apresentado por meio de pesquisa direta em sites e vídeos na internet, por análises de documentos, de jornais e pesquisas realizadas por pesquisadores e alunos, sempre sob a perspectiva sociológica. \\

O local observado do trabalho será a Grande Goiânia e suas regiões metropolitanas e os fatos serão o lixo como resultado direto da ação humana. Portanto, terá a articulação entre a teoria e a prática , que consistirá na observação e identificação dos problemas sociais que afetam a qualidade de vida da população, e como o contraste comparativo presente em áreas da cidade e como isso passa despercebido pela sociedade. \\

Ademais, vale considerar que a pesquisa teve que ser feita sem ir a campo, em virtude do risco de contaminação pelo vírus da COVID 19.