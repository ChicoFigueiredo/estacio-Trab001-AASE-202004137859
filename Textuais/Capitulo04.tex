\chapter{Resultados e Conclusão}

Nas palavras de \citeaa{Magalhaes2018} "A educação ambiental impacta não apenas no meio em que vivemos, mas está diretamente ligada à
sobrevivência humana, e precisa estar presente no ensino de forma incisiva.
A introdução da educação ambiental nos primeiros anos da educação infantil potencializa o
processo de ensino-aprendizagem, uma vez que o ambiente escolar é um dos meios de integração e
conscientização mais completos para abordar as problemáticas entre a relação homem e natureza.
Quando a educação ambiental é aplicada desde o início do processo de educação e se torna
constante nos anos subsequentes, a aprendizagem transforma-se permanentemente.
É evidente que as mudanças no meio ambiente ocorrem de forma lenta e gradativa, mas quanto
antes iniciado o processo de educação e conscientização da população, maiores são as chances de
sucesso. Assim, é de fato extremamente importante que a Educação Ambiental seja inserida desde
os primeiros anos da educação infantil.
Entretanto, este não é um dever apenas da escola: é fundamental que todos os segmentos da
sociedade em que a criança está inserida se envolvam e busquem este objetivo comum. Está
conscientização das crianças também é um dever dos pais e da sociedade em geral."\\


