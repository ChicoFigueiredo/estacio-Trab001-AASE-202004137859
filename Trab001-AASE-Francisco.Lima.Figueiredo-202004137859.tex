\documentclass[
	12pt,				% tamanho da fonte
	openright,			% capítulos começam em pág ímpar (insere página vazia caso preciso)
	twoside,			% para impressão em recto e verso. Oposto a oneside
	a4paper,			% tamanho do papel.
	chapter=TITLE,		% títulos de capítulos convertidos em letras maiúsculas
	section=TITLE,		% títulos de seções convertidos em letras maiúsculas
	sumario=abnt-6027-2012,
	english,			% idioma adicional para hifenização
	brazil				% o último idioma é o principal do documento
]{abntex2}
\usepackage{import}    % package import tem o comando import que faz a importação com "novo path"
\usepackage{packages/udescCCT} 	% Pacote de customizações da UDESC/CCT

\import{0-conf/}{!main}

% -----------------------------------------------------------------
% Informações de dados para CAPA e FOLHA DE ROSTO
% -----------------------------------------------------------------
\tipotrabalho{Atividade prática como componente curricular}
\titulo{Observações e análise sociológica reflexiva das relações entre a sociedade e o meio ambiente}%
% ATENÇÃO: O símbolo {} indica o sobrenome para a ficha catalográfica.
% Exemplo: Sherlock Holmes {}da Silva para sobrenomes compostos;
% Exemplo: Arnold Alois {}Schwarzenegger para sobrenome simples.
\autor{Francisco Lima {}Figueiredo}%
\orientador{Fernando de {}Figueiredo Balieiro}%
\coorientador{Daniel Tadeu {}do Amaral}%
\instituicao{Universidade Estácio de Sá}%
\preambulo{Trabalho apresentada ao professor Daniel Tadeu do Amaral como parte dos trabalhos a serem apresentados na disciplina ASPECTOS ANTROPOLÓGICOS E SOCIOLÓGICOS DA EDUCAÇÃO (CEL0466/3521060 - 9011).}
\local{Brasília}%
\data{\the\year}%
% ---

% compila o indice
\makeindex

% -----------------------------------------------------------------
% Início do documento
% -----------------------------------------------------------------
\begin{document}
	\import{1-pream/}{!main}

    \import{2-caps/}{!main}

    \import{4-pos/}{!main}

    % -----------------------------------------------------------------
    % ELEMENTOS PÓS-TEXTUAIS
    % -----------------------------------------------------------------
    \postextual

    % Você pode comentar os elementos que não deseja em seu trabalho;

    % Referências bibliográficas
    %\bibliography{PosTextuais/Biblio-Trab0001}	    % Elemento Obrigatório
    \bibliography{PosTextuais/Biblio-Trab0001}	    % Elemento Obrigatório

    \include{PosTextuais/Glossario}			    % Elemento Opcional
    \include{PosTextuais/Apendices}				% Elemento Opcional
    
% ----------------------------------------------------------
% Anexos
% ----------------------------------------------------------
%
% ---
% Inicia os anexos
% ---
\begin{anexosenv}

% Imprime uma página indicando o início dos anexos
\partanexos

Documento programado em \LaTeX  e seu projeto está disponível no endereço (legível pelo QR Code abaixo): \\ \href{https://github.com/ChicoFigueiredo/estacio-Trab001-AASE-Francisco.Lima.Figueiredo-202004137859.git}{ https://github.com/ChicoFigueiredo/estacio-Trab001-AASE-Francisco.Lima.Figueiredo-202004137859.git }\\

\qrset{link, height=4cm}
\qrcode{https://github.com/ChicoFigueiredo/estacio-Trab001-AASE-Francisco.Lima.Figueiredo-202004137859.git}

\end{anexosenv}
				% Elemento Opcional
    %\include{PosTextuais/IndiceRemissivo}		% Elemento Opcional

\end{document}

% -----------------------------------------------------------------
% Fim do Documento
% -----------------------------------------------------------------